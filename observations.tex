\chapter{Observations and analyses}\label{chap:observations}
To evaluate the validity of the above theory, I have developed and used analysis software
available \emph{https://github.com/mkosunen/Covid19\_SydeKick}. Software
fetches the data automatically from Johns Hopkins database and plots the
relative growth and the number of current cases for chosen countries under
interest. Observations of the growth and number of cases will be to compared
to the presented theory so the theory can be adjusted accordingly, or the
root causes /explanations for the anomalies will be developed. 

\section{Action log}
To this section I try to log the implemented governmental actions that
supposedly decrerase $S$. These actions can be compared to data to evaluate
the effectiveness of the actions. A quite comprehensive list of actions in
Finalnd is presented in
\emph{https://timoheinonen.fi/koronapaivakirja-nain-covid-19-mullisti-maailman}
.

\section{Observations}

\begin{figure}
    \centering
    \includegraphics[width=\textwidth]{Figures/Analysis_figures/Covid19_in_Finland.png}
    \caption{Relative growth and number of cases in Finland.}\label{fig:Finland}
\end{figure}

\begin{figure}
    \centering
    \includegraphics[width=\textwidth]{Figures/Analysis_figures/Covid19_in_Sweden.png}
    \caption{Relative growth and number of cases in Sweden.}\label{fig:Sweden}
\end{figure}

\begin{figure}
    \centering
    \includegraphics[width=\textwidth]{Figures/Analysis_figures/Covid19_in_Norway.png}
    \caption{Relative growth and number of cases in Norway.}\label{fig:Norway}
\end{figure}

\begin{figure}
    \centering
    \includegraphics[width=\textwidth]{Figures/Analysis_figures/Covid19_in_Denmark.png}
    \caption{Relative growth and number of cases in Denmark.}\label{fig:Denmark}
\end{figure}
\begin{figure}
    \centering
    \includegraphics[width=\textwidth]{Figures/Analysis_figures/Covid19_in_Italy.png}
    \caption{Relative growth and number of cases in Italy.}\label{fig:Italy}
\end{figure}

\begin{figure}
    \centering
    \includegraphics[width=\textwidth]{Figures/Analysis_figures/Covid19_in_Germany.png}
    \caption{Relative growth and number of cases in Germany.}\label{fig:Germany}
\end{figure}

\begin{figure}
    \centering
    \includegraphics[width=\textwidth]{Figures/Analysis_figures/Covid19_in_France.png}
    \caption{Relative growth and number of cases in France.}\label{fig:France}
\end{figure}

\begin{figure}
    \centering
    \includegraphics[width=\textwidth]{Figures/Analysis_figures/Covid19_in_Spain.png}
    \caption{Relative growth and number of cases in Spain.}\label{fig:Spain}
\end{figure}

\begin{figure}
    \centering
    \includegraphics[width=\textwidth]{Figures/Analysis_figures/Covid19_in_US.png}
    \caption{Relative growth and number of cases in United States.}\label{fig:US}
\end{figure}

\begin{figure}
    \centering
    \includegraphics[width=\textwidth]{Figures/Analysis_figures/Covid19_in_Korea, South.png}
    \caption{Relative growth and number of cases in South Korea.}\label{fig:SouthKorea}
\end{figure}

\begin{figure}
    \centering
    \includegraphics[width=\textwidth]{Figures/Analysis_figures/Covid19_in_China.png}
    \caption{Relative growth and number of cases in China.}\label{fig:China}
\end{figure}

\begin{figure}
    \centering
    \includegraphics[width=\textwidth]{Figures/Analysis_figures/Covid19_Selected_cases.png}
    \caption{Relative growth and number of cases in Selected countries.}\label{fig:Selected}
\end{figure}
